\documentclass[11pt,a4paper]{article}
\usepackage[utf8]{inputenc}
\usepackage[T1]{fontenc}
\usepackage[margin=1in]{geometry}
\usepackage{booktabs}
\usepackage{longtable}
\usepackage{array}
\usepackage{enumitem}
\usepackage{xcolor}
\usepackage{hyperref}
\usepackage{graphicx}
\usepackage{fancyhdr}
\usepackage{titlesec}
\usepackage{titletoc}

\hypersetup{
    colorlinks=true,
    linkcolor=blue,
    filecolor=magenta,      
    urlcolor=cyan,
}

% Customize page style - no header/footer on first pages
\fancypagestyle{plain}{%
    \fancyhf{}%
    \renewcommand{\headrulewidth}{0pt}%
}

\pagestyle{fancy}
\fancyhf{}
\fancyhead[L]{PPIT Final Exam - Combined Cheat Sheet}
\fancyhead[R]{\thepage}
\renewcommand{\headrulewidth}{0.4pt}

\setlength{\parindent}{0pt}
\setlength{\parskip}{6pt}

% Set table of contents depth (0=parts, 1=sections, 2=subsections, 3=subsubsections)
\setcounter{tocdepth}{1}

% Format title page
\title{\vspace{-2cm}\Huge\textbf{PPIT Final Exam}\\[0.5cm]\Large Combined Cheat Sheet\vspace{1cm}}
\author{\Large\textbf{Exam Information}\\[0.3cm]
        \large Exam Date: December 26th, 2024\\[0.2cm]
        Total Marks: 100\\[0.2cm]
        Time: 3 Hours\vspace{1.5cm}}
\date{}

% Format table of contents
\titlecontents{part}
    [0pt]
    {\addvspace{10pt}\bfseries\large}
    {\contentslabel{0pt}}
    {}
    {\titlerule*[0.5pc]{.}\contentspage}
    []

\titlecontents{section}
    [20pt]
    {\addvspace{2pt}}
    {\contentslabel{2.5em}}
    {}
    {\titlerule*[0.5pc]{.}\contentspage}
    []

\begin{document}

% Title page
\thispagestyle{plain}
\maketitle
\vfill
\begin{center}
\large\textit{Comprehensive Study Guide for PPIT Final Examination}
\end{center}
\vspace{2cm}

\newpage

% Table of Contents page
\thispagestyle{plain}
\begin{center}
\vspace*{1cm}
{\Huge\textbf{Table of Contents}}\\[0.5cm]
\rule{0.8\textwidth}{0.4pt}\\[1cm]
\end{center}
\addcontentsline{toc}{section}{Table of Contents}
\tableofcontents
\vfill
\begin{center}
\rule{0.8\textwidth}{0.4pt}
\end{center}

\newpage

% Exam Structure Overview page
\thispagestyle{plain}
\begin{center}
\vspace*{1cm}
{\Huge\textbf{Exam Structure Overview}}\\[0.5cm]
\rule{0.8\textwidth}{0.4pt}\\[1cm]
\end{center}

\vspace{0.5cm}
\addcontentsline{toc}{section}{Exam Structure Overview}

\begin{center}
\begin{tabular}{p{2.5cm}p{2cm}p{4.5cm}p{5cm}}
\toprule
\textbf{Question} & \textbf{Marks} & \textbf{Type} & \textbf{Focus Areas} \\
\midrule
\textbf{Q1} & 25 & Multiple Choice (25 questions) & Broad coverage of all topics \\
\midrule
\textbf{Q2} & 25 & Ethical Decision Making (4-step process) & Ethics, ethical theories, stakeholder analysis \\
\midrule
\textbf{Q3} & 15 & Business Structures & Veil of incorporation, company structures \\
\midrule
\textbf{Q4} & 15 & Short Answer Questions & Licensing, computer crimes, contracts \\
\midrule
\textbf{Q5} & 10 & Communication & Grapevine communication, rumor management \\
\midrule
\textbf{Q6} & 10 & Organizational Structure & Designing organizational structure \\
\bottomrule
\end{tabular}
\end{center}

\vspace{1.5cm}

\begin{center}
\rule{0.8\textwidth}{0.4pt}\\[0.5cm]
\textit{Total: 100 Marks | Duration: 3 Hours}
\end{center}

\vfill

\part{HIGH PRIORITY TOPICS (50+ Marks)}

\section{ETHICAL THEORIES \& 4-STEP DECISION MAKING (Q2 - 25 marks)}

\subsection{Core Ethical Theories}

\begin{longtable}{p{3cm}p{4cm}p{3cm}p{2.5cm}}
\toprule
\textbf{Theory} & \textbf{Key Principle} & \textbf{Focus} & \textbf{Key Figure} \\
\midrule
\textbf{Utilitarianism} & Greatest good for greatest number & Outcomes, consequences & Consequentialist \\
\textbf{Deontology} & Duty-based, categorical imperative & Rules and duties, not consequences & Kant \\
\textbf{Egoism} & Self-interest based & Personal benefit & NOT duty-based \\
\textbf{Hedonism} & Pursuit of happiness/pleasure & Pleasure as highest aim & -- \\
\textbf{Consequentialism} & Focus on outcomes & Results matter most & Utilitarianism is a type \\
\textbf{Rights \& Duties} & Individual rights + moral duties & Fairness, fundamental rights & -- \\
\textbf{Virtue Ethics} & Character and virtues & ``What would a virtuous person do?'' & -- \\
\textbf{Justice/Fairness} & Fair treatment and equity & Distribution of benefits/harms & -- \\
\bottomrule
\end{longtable}

\subsection{4-Step Ethical Decision Making Framework}

\subsubsection{Step I: Understanding the Situation}
\begin{enumerate}
\item \textbf{List relevant facts} (10--15 facts, numbered)
\begin{itemize}
\item Neutral, logical exercise
\item Do NOT judge at this stage
\item Just state what is known
\end{itemize}
\item \textbf{Identify ethical issues} (4--6 issues)
\begin{itemize}
\item Why it's an ethical issue
\item Potential or resulting harm
\item Look for: discrimination, privacy violations, safety risks, conflicts of interest
\end{itemize}
\item \textbf{List stakeholders} (8--10 stakeholders)
\begin{itemize}
\item Directly and indirectly affected parties
\item Think broadly: applicants, employees, management, customers, society, industry, regulators
\end{itemize}
\end{enumerate}

\subsubsection{Step II: Isolating the Major Ethical Dilemma}
\begin{itemize}
\item Format: ``Should [person/entity] [action] or [alternative action]?''
\item Identify core ethical conflict
\item What values/rights/duties compete?
\item Frame as clear either/or question
\end{itemize}

\subsubsection{Step III: Analyzing Ethicality (BOTH ALTERNATIVES)}

\textbf{A. CONSEQUENTIALISM (Utilitarian Analysis)}
\begin{itemize}
\item A. If action done, who harmed?
\item B. If action NOT done, who harmed?
\item C. Which preferable, A or B?
\item D. If action done, who benefits?
\item E. If action NOT done, who benefits?
\item F. Which preferable, D or E?
\item G. Which produces greatest good for greatest number?
\item Consider both short-term and long-term consequences
\end{itemize}

\textbf{B. RIGHTS AND DUTIES}
\begin{enumerate}
\item What rights do stakeholders have?
\item What duties do we have toward stakeholders?
\item Which rights/duties take precedence when they conflict?
\item Does action respect fundamental rights?
\item Does action fulfill our duties?
\end{enumerate}
\begin{itemize}
\item Fundamental rights (fairness, non-discrimination) usually take precedence over efficiency
\end{itemize}

\textbf{C. DEONTOLOGY (Kant's Categorical Imperative)}
\begin{itemize}
\item H. If action done, who treated with disrespect?
\item I. If action NOT done, who treated with disrespect?
\item J. Which preferable, H or I?
\item K. If action done, who treated unlike others?
\item L. If action NOT done, who treated unlike others?
\item M. Which preferable, K or L?
\item N. Benefits if everyone did this action?
\item O. Benefits if nobody did this action?
\item P. Which preferable, N or O?
\item Universalization test: Would it be okay if everyone acted this way?
\end{itemize}

\subsubsection{Step IV: Making Decision and Implementation}
\begin{enumerate}
\item \textbf{Make defensible ethical decision}
\begin{itemize}
\item Reference supporting frameworks (A--G, 1--5, H--P)
\item Justify choice
\item Explain which rights/duties take precedence
\item Be clear and direct: ``Should/Should not\ldots''
\end{itemize}

\item \textbf{List implementation steps} (5--8 concrete steps)
\begin{itemize}
\item Document findings
\item Escalate to management
\item Suspend problematic activity
\item Conduct audit/review
\item Retrain staff
\item Implement oversight
\item Communicate with stakeholders
\item Be specific, not vague
\end{itemize}

\item \textbf{Show stakeholder impacts}
\begin{itemize}
\item For each stakeholder: positive impacts, benefits, challenges
\item Go through each stakeholder from Step I.C
\item Be balanced -- acknowledge both benefits and challenges
\end{itemize}

\item \textbf{Longer-term preventive changes} (4--6 measures)
\begin{itemize}
\item \textbf{Technical}: Bias testing, audits, standards
\item \textbf{Organizational}: Ethics codes, review boards, training, culture
\item \textbf{Legal/Compliance}: Legislation, industry standards, regulations
\item \textbf{Cultural}: Ethical engineering culture, open discussion
\item \textbf{Societal}: Public awareness, education, diversity
\item \textbf{Industry}: Collaboration, best practices, professional organizations
\end{itemize}

\item \textbf{Pivot point prevention}
\begin{itemize}
\item What should have been done at design/planning stage?
\item Ethical impact assessment, diverse data, bias testing
\item Cross-functional teams, early intervention
\end{itemize}
\end{enumerate}

\subsection{Key Ethics Definitions}
\begin{itemize}
\item \textbf{Whistleblowers}: Make unauthorized disclosures about harmful situations or fraud
\item \textbf{Moral principles $\neq$ Law}: They can differ (False that they always align)
\item \textbf{Code of conduct objectives}: Discipline, Inspiration, Education (NOT Enforcement)
\item \textbf{Ethical theories are NOT formulas}: They are frameworks for analysis (NOT formulas to solve problems, NOT for judging involuntary actions)
\item \textbf{Egoism is NOT duty-based}: Egoism is self-interest based, NOT based on altruism or care for others
\item \textbf{ACM Code Requirement}: Give comprehensive and thorough evaluations of computer systems and their impacts, including analysis of possible risks (True)
\item \textbf{Utilitarianism is consequentialist}: Focuses on outcomes and collective welfare
\item \textbf{Deontology is attributed to Kant}: Duty-based ethical theory
\item \textbf{Profession traits}: Expert knowledge, substantial education/training, autonomy, internal governance, service to society, codes of conduct, professional bodies
\item \textbf{Mature profession}: Education + accreditation + certification + licensing + CPD + codes + societies
\item \textbf{Pillars of Professionalism}: Commitment, Integrity, Responsibility, Accountability
\end{itemize}

\subsection{Informal Guidelines (Quick Tests)}
\begin{enumerate}
\item \textbf{Secrecy Test} -- Is anyone asking me to keep this quiet?
\item \textbf{Mom Test} -- Would I be proud to tell my mother?
\item \textbf{TV Test} -- Would I be comfortable if this appeared on news?
\item \textbf{Market Test} -- Could this be advertised as a selling point?
\item \textbf{Smell Test} -- Does something feel ``off'' instinctively?
\end{enumerate}

\subsection{Golden Rule}
Would I accept this if roles were reversed?

\section{PROFESSIONAL CODE OF CONDUCT}

\subsection{ACM Code of Ethics - General Principles}
\begin{enumerate}
\item Contribute to society and human well-being
\item Avoid harm
\item Be honest and trustworthy
\item Be fair and non-discriminatory
\item Respect intellectual property
\item Respect privacy
\item Honor confidentiality
\end{enumerate}

\subsection{Professional Responsibilities}
\begin{itemize}
\item Achieve high quality in processes and products
\item Maintain professional competence
\item Know and respect existing rules
\item Accept professional review
\item \textbf{Give comprehensive and thorough evaluations} of computer systems and their impacts, including analysis of possible risks
\item Honor contracts and agreements
\item Improve public understanding
\item Access resources only when authorized
\end{itemize}

\subsection{Leadership Principles}
\begin{itemize}
\item Ensure public good is central concern
\item Articulate and evaluate social responsibilities
\item Manage resources to enhance quality of working life
\item Create growth opportunities
\item Care when modifying/retiring systems
\end{itemize}

\subsection{IEEE Code of Ethics - Key Principles}
\begin{itemize}
\item Accept responsibility for safety, health, welfare of public
\item Avoid conflicts of interest
\item Be honest in claims/estimates
\item Reject bribery
\item Improve understanding of technology
\item Maintain technical competence
\item Seek and offer honest criticism
\item Treat all persons fairly
\item Avoid injuring others
\item Assist colleagues' professional development
\end{itemize}

\subsection{Ethical Issues in Computing}
\begin{itemize}
\item \textbf{Privacy and Data Protection}: Collect only necessary, secure, informed consent
\item \textbf{Security}: Robust measures, responsible disclosure, protect from unauthorized access
\item \textbf{Intellectual Property}: Respect copyrights, patents, trademarks, proper attribution
\item \textbf{Accessibility}: Design for disabilities, follow WCAG, Section 508
\item \textbf{Quality and Reliability}: Thorough testing, honest reporting, proper maintenance
\end{itemize}

\subsection{Whistleblowing}
\begin{itemize}
\item \textbf{Definition}: Reporting unethical, illegal, or harmful activities
\item \textbf{When Justified}: Clear evidence, internal channels exhausted, significant harm, proper motives, proportional disclosure
\item \textbf{Risks}: Retaliation, career damage, legal consequences, personal/financial stress, social isolation
\item \textbf{Protections}: Legal protections (varies by jurisdiction), professional code support, internal policies, anonymous reporting
\end{itemize}

\subsection{Conflict of Interest}
\begin{itemize}
\item \textbf{Types}: Financial, personal relationships, competing loyalties, gifts/favors, outside employment
\item \textbf{Managing}: Disclose, recuse, avoid, follow policies, seek guidance
\item \textbf{Examples in Computing}: Working for competitor, using company resources for personal projects, accepting vendor gifts, hiring friends/family improperly
\end{itemize}

\subsection{Professional Liability}
\begin{itemize}
\item \textbf{Types}: Legal responsibility for actions, malpractice in services, negligence in design/implementation, product liability for defects
\item \textbf{Limiting Liability}: Professional liability insurance, clear contracts, proper documentation, following industry standards, regular professional development
\item \textbf{Responsibility Chain}: Designers, Developers, Testers, Managers, Organizations all share responsibility
\end{itemize}

\section{ISLAMIC ETHICS}

\subsection{Foundational Principles}
\begin{itemize}
\item \textbf{Tawheed} (Oneness): Belief in absolute oneness of Allah
\item \textbf{Accountability} (Hisab): Belief in accountability before Allah
\item \textbf{Justice} (Adl): Fairness and equity in all dealings
\item \textbf{Trust} (Amanah): Fulfilling responsibilities and obligations
\item \textbf{Truthfulness} (Sidq): Honesty in words and actions
\item \textbf{Compassion} (Rahmah): Showing mercy and kindness
\item \textbf{Moderation} (Wasatiyyah): Avoiding extremes, balanced approach
\end{itemize}

\subsection{Core Ethical Values}
\begin{itemize}
\item Honesty and Integrity, Justice and Fairness, Compassion and Mercy
\item Respect and Dignity, Responsibility, Humility, Patience (Sabr)
\item Gratitude (Shukr), Forgiveness, Generosity
\end{itemize}

\subsection{Sources of Islamic Ethics}
\begin{enumerate}
\item \textbf{Quran (Holy Book)}: Primary source of Islamic guidance
\item \textbf{Sunnah (Prophetic Tradition)}: Teachings, actions, and approvals of Prophet Muhammad (peace be upon him)
\item \textbf{Ijma (Consensus)}: Agreement of Islamic scholars
\item \textbf{Qiyas (Analogical Reasoning)}: Applying established principles to new situations
\item \textbf{Ijtihad (Independent Reasoning)}: Scholarly effort to derive rulings for new situations
\end{enumerate}

\subsection{Islamic Business Ethics}
\begin{itemize}
\item \textbf{Lawful Earnings (Halal)}: Earning through permissible means only
\item \textbf{Honesty in Trade}: Truthful representation, fair pricing, full disclosure
\item \textbf{Fair Contracts}: Clear agreements, mutual consent, fulfilling obligations
\item \textbf{Prohibition of Riba}: Avoiding interest-based transactions
\item \textbf{Prohibition of Gharar}: Avoiding excessive uncertainty
\item \textbf{Fair Treatment of Employees}: Fair wages, safe conditions, respect
\item \textbf{Environmental Responsibility}: Protecting environment, sustainable practices
\end{itemize}

\subsection{Islamic Professional Ethics}
\begin{itemize}
\item Professional Competence, Work Ethics (punctuality, reliability, diligence)
\item Honesty and Integrity, Fair Treatment, Responsibility
\item Respect for Authority, Teamwork and Cooperation, Avoiding Harm
\end{itemize}

\subsection{Islamic Ethics in Technology and IT}
\begin{itemize}
\item Ethical Use: Using technology for beneficial purposes, avoiding harm
\item Privacy and Data Protection: Respecting privacy, protecting data, maintaining confidentiality
\item Intellectual Property: Respecting IP rights, not engaging in piracy, giving credit
\item Honest Representation: Accurate representation, not misleading users
\item Avoiding Harmful Content: Not creating/distributing harmful content
\item Fair Access: Ensuring fair access, not discriminating, promoting digital inclusion
\item Environmental Considerations: Considering environmental impact, sustainable practices
\item Social Responsibility: Using technology for social good, ethical innovation
\end{itemize}

\section{HUMAN RIGHTS}

\subsection{Core Principles}
\begin{itemize}
\item \textbf{Universality}: Rights for all
\item \textbf{Inalienability}: Cannot be taken away
\item \textbf{Equality/Non-discrimination}: Equal treatment
\item \textbf{Accountability}: Duty-bearers responsible
\item \textbf{Rule of Law}: Legal framework
\end{itemize}

\subsection{Key Instruments}
\begin{itemize}
\item \textbf{UDHR (1948)}: Art.1 (dignity/equality), Art.2 (non-discrimination), Art.3 (life/liberty), Art.12 (privacy), Art.19 (expression), Art.23/25 (work, adequate living)
\item \textbf{ICCPR / ICESCR}: Civil-political vs. economic-social-cultural rights
\item \textbf{Pakistan Constitution}: Arts. 8--28 (fundamental rights), Art.19 (speech w/ restrictions), Art.19A (right to information), Art.25 (equality)
\end{itemize}

\subsection{Categories (``Generations'') of Rights}
\begin{itemize}
\item \textbf{1st}: Civil \& political (life, liberty, fair trial, expression, privacy)
\item \textbf{2nd}: Economic, social, cultural (work, education, health, housing)
\item \textbf{3rd}: Collective/solidarity (development, environment, self-determination)
\end{itemize}

\subsection{Duties \& Duty-Bearers}
\begin{itemize}
\item \textbf{State}: Respect, Protect, Fulfill
\item \textbf{Corporate (UNGPs)}: Respect rights, human-rights due diligence, enable remedy
\item \textbf{Professionals}: Avoid complicity; uphold privacy/fairness/security
\end{itemize}

\subsection{Tech/Digital Rights Focus}
\begin{itemize}
\item \textbf{Privacy \& Data}: Minimization, consent, purpose limitation, security, retention limits; risks---surveillance, breaches, secondary use, re-ID
\item \textbf{Expression \& Information}: Art.19/19A; ensure due process for takedown/appeal; proportional restrictions
\item \textbf{Non-discrimination}: Algorithmic bias $\rightarrow$ need fairness, explainability, bias audits
\item \textbf{Due process}: Notice and ability to contest automated decisions
\item \textbf{Access \& Inclusion}: Digital divide, affordability, accessibility (WCAG), localization
\item \textbf{Safety/Security}: Cybersecurity as an enabler of rights; protection from harassment/stalking
\end{itemize}

\subsection{Limitations on Rights}
\begin{itemize}
\item Must be \textbf{lawful, necessary, proportionate, for legitimate aim} (e.g., security, public order, rights of others)
\item No arbitrary/broad restrictions
\end{itemize}

\subsection{Business \& Human Rights (Tech)}
\begin{itemize}
\item \textbf{UNGPs}: Protect--Respect--Remedy; due diligence; grievance mechanisms
\item Areas: content moderation, surveillance tech, ad targeting, AI/ML bias, supply chain labor, transparency reports
\end{itemize}

\section{BUSINESS ORGANIZATIONS \& STRUCTURES (Q3 - 15 marks)}

\subsection{Veil of Incorporation}
\begin{itemize}
\item \textbf{Definition}: Legal principle separating company from owners (limited liability)
\item \textbf{Purpose}: Protects owners' personal assets from company debts
\end{itemize}

\subsection{Business Structure Comparison}

\begin{longtable}{p{2.5cm}p{2.5cm}p{2.5cm}p{2.5cm}p{2.5cm}p{2.5cm}}
\toprule
\textbf{Feature} & \textbf{Sole Proprietorship} & \textbf{Partnership/AOP} & \textbf{Public Ltd.} & \textbf{SMC Pvt. Ltd.} & \textbf{Private Ltd.} \\
\midrule
\textbf{Owners} & 1 & 2+ (no upper limit) & 7+ (unlimited) & 1 & 2--100 \\
\textbf{Directors} & N/A & N/A & 3+ & 1 (must nominate 2 alternates) & 2+ \\
\textbf{Legal Entity} & No (not separate) & No (not separate) & Yes & Yes & Yes \\
\textbf{Liability} & Unlimited & Unlimited (joint and several) & Limited & Limited & Limited \\
\textbf{Suffix} & None & None & Limited (Ltd.) & (SMC-Private) Limited & Private Limited (Pvt Ltd) \\
\textbf{Prospectus} & No & No & Required & Not allowed & Not required \\
\textbf{Public Subscription} & No & No & Allowed & Not allowed & Not allowed \\
\textbf{Company Secretary} & No & No & Required & Required (sole director cannot be) & Optional \\
\textbf{Taxation} & Personal income tax & AOP pays tax once & Corporate tax & Corporate tax & Corporate tax \\
\textbf{Registration} & FBR (if income threshold) & FBR + Registrar of Firms & SECP & SECP & SECP \\
\bottomrule
\end{longtable}

\subsection{Employment Contract Clauses}
\begin{itemize}
\item \textbf{Non-compete Clause}: Prevents employees (current AND former) from working for direct competitors of a business for a period of time
\item \textbf{Non-solicitation Clause}: Prevents employees from encouraging other employees AND/OR organization's customers to move to another company
\end{itemize}

\subsection{Key Requirements}
\begin{itemize}
\item \textbf{SMC}: Must nominate 2 individuals (one nominee director, one alternate); must appoint company secretary (sole director cannot be secretary)
\item \textbf{Public Ltd.}: 7+ members, 3+ directors, prospectus required, public subscription allowed
\item \textbf{Private Ltd.}: 2--100 shareholders, 2+ directors, not publicly traded
\item \textbf{Partnership/AOP}: Partnership Deed defines ownership, profit/loss division, decision-making; registered with FBR and Registrar of Firms
\end{itemize}

\subsection{Partnership Deed (10 Key Provisions)}
\begin{enumerate}
\item Firm name and business type
\item Partners' names and addresses
\item Capital contribution by each partner
\item Profit and loss sharing ratio
\item Decision-making powers
\item Rules for adding/removing partners
\item Handling partner's death/withdrawal
\item Duration (fixed term or indefinite)
\item Banking arrangements
\item Dissolution procedure
\end{enumerate}

\subsection{Types of Organizations}
\begin{itemize}
\item \textbf{Public Sector}: Owned and operated by government (NADRA, PIA, State Bank of Pakistan)
\item \textbf{Private Sector}: Owned by individuals or private groups (DPL, Systems Ltd, TPL Corp)
\item \textbf{Non-Profit (NPO)}: Operates for social welfare (Edhi Foundation, Shaukat Khanum)
\item \textbf{International}: Operate across borders (UNDP Pakistan, WHO)
\end{itemize}

\subsection{Key Considerations in Choosing Business Structure}
\begin{enumerate}
\item Liability Protection: Need for limited liability vs. personal responsibility
\item Tax Implications: Tax treatment, planning opportunities, compliance
\item Capital Requirements: Ability to raise capital, investment needs
\item Management and Control: Desired control level, decision-making structure
\item Regulatory Compliance: Reporting requirements, compliance costs
\item Business Goals: Growth plans, exit strategy, long-term objectives
\item Number of Owners: Single vs. multiple owners, partnership considerations
\end{enumerate}

\section{ORGANIZATIONAL STRUCTURE \& DESIGN (Q6 - 10 marks)}

\subsection{Types of Organizational Structures}

\begin{longtable}{p{2.5cm}p{4cm}p{3cm}p{3cm}}
\toprule
\textbf{Type} & \textbf{Description} & \textbf{Best For} & \textbf{Example} \\
\midrule
\textbf{Simple} & Few layers, central decision-maker & Small startups & Small teams \\
\textbf{Functional} & Organized by departments (HR, Tech, Sales) & Mid-sized companies & Department-based \\
\textbf{Divisional/Product} & Based on products/clients & Product-focused orgs & Product teams \\
\textbf{Matrix} & Dual reporting (functional + project) & Complex projects & Consulting firms \\
\textbf{Flat} & Minimal hierarchy, wide control span & Innovation-focused & DPL, Netflix \\
\textbf{Hierarchical/Tall} & Many levels, clear authority chain & Traditional corporates & Banks, NADRA \\
\textbf{Network} & Relies on external partners & Outsourcing models & Virtual companies \\
\textbf{Team-based} & Built around teams & Agile organizations & Cross-functional teams \\
\bottomrule
\end{longtable}

\subsection{Key Elements of Structure}

\begin{longtable}{p{3.5cm}p{5cm}p{5cm}}
\toprule
\textbf{Element} & \textbf{Meaning} & \textbf{Example} \\
\midrule
\textbf{Work Specialization} & Division of tasks into specific jobs & Separate DevOps, Design, Sales \\
\textbf{Departmentalization} & Grouping jobs (function, product, geography) & IT Dept., Product Teams \\
\textbf{Chain of Command} & Who reports to whom & CEO $\rightarrow$ Managers $\rightarrow$ Teams \\
\textbf{Span of Control} & Number of people manager supervises & Wide = autonomy; Narrow = control \\
\textbf{Centralization} & Where decisions made & Centralized vs. Decentralized \\
\textbf{Formalization} & Degree of rules and procedures & High = banks; Low = startups \\
\bottomrule
\end{longtable}

\subsection{Factors Affecting Structure}

\begin{longtable}{p{2.5cm}p{4cm}p{4cm}}
\toprule
\textbf{Factor} & \textbf{Impact} & \textbf{Example} \\
\midrule
\textbf{Strategy} & Structure must support goals & Innovation $\rightarrow$ flat; Control $\rightarrow$ hierarchical \\
\textbf{Size} & Larger = more specialization & 10 people = simple; 10,000 = complex \\
\textbf{Technology} & Complexity affects coordination & Software = agile teams; Manufacturing = process-based \\
\textbf{Environment} & Stable vs. dynamic & Government = stable; Tech = dynamic \\
\textbf{Culture \& Leadership} & Values shape structure & People-first $\rightarrow$ flat; Control $\rightarrow$ tall \\
\textbf{Geography} & Multiple regions $\rightarrow$ divisional & Multinationals = geographic divisions \\
\textbf{Legal/Regulatory} & Some sectors require formal hierarchies & Regulated industries = layered \\
\bottomrule
\end{longtable}

\subsection{Tall vs. Flat Organizations}

\begin{longtable}{p{2.5cm}p{5cm}p{5cm}}
\toprule
\textbf{Feature} & \textbf{Tall} & \textbf{Flat} \\
\midrule
\textbf{Levels} & Many & Few \\
\textbf{Decision Speed} & Slow & Fast \\
\textbf{Control} & Tight supervision & Empowered teams \\
\textbf{Communication} & Top-down & Open and horizontal \\
\textbf{Example} & NADRA, Banks & DPL, Netflix \\
\textbf{Characteristics} & Bureaucratic, hierarchical, formal & Flexible, informal, empowered \\
\bottomrule
\end{longtable}

\subsection{For Q6 (Designing Structure):}
\begin{itemize}
\item Analyze: Products, geography, centralized functions
\item Suggest: Likely Matrix or Hybrid structure
\item Justify: Based on strategy, size, technology, geography
\item Draw: Organization chart showing reporting relationships
\end{itemize}

\subsection{Example Scenario: Kangaroo Corp (Multi-product, Multi-geography)}
\begin{itemize}
\item \textbf{Products}: Multiple software packages (Farming Basics, Premium, Analytics)
\item \textbf{Geography}: Multiple countries (Australia, USA, Pakistan)
\item \textbf{Centralized Functions}: Software development, customer services, Technology (CTO), PMO
\item \textbf{Country Operations}: Business development, sales, marketing at country level
\item \textbf{Structure Recommendation}: Matrix or Hybrid (product divisions + geographic divisions + centralized functions)
\item \textbf{Justification}: Need to balance product focus, geographic responsiveness, and centralized efficiency
\item \textbf{Key Elements}: 
\begin{itemize}
\item Product divisions (FB, FP, FA) with flat hierarchy
\item Geographic sub-offices (USA, Pakistan) for country operations
\item Centralized head office (Sydney) for development and customer services
\item Centralized Technology (CTO) and PMO for technical/project management
\end{itemize}
\end{itemize}

\section{ORGANIZATIONAL CULTURE}

\subsection{What Is Organizational Culture?}
\begin{itemize}
\item \textbf{Definition}: The shared system of values, beliefs, norms, and behaviors that shapes how people within an organization think, feel, and act
\item \textbf{Metaphor}: The ``personality'' of an organization
\end{itemize}

\subsection{Components}
\begin{itemize}
\item \textbf{Values}: What the organization considers important
\item \textbf{Beliefs}: Assumptions about how things work
\item \textbf{Norms}: Unwritten rules of behavior
\item \textbf{Behaviors}: Observable actions and practices
\item \textbf{Symbols}: Visible representations of culture
\end{itemize}

\subsection{What Do Cultures Do?}
\begin{enumerate}
\item \textbf{Defines the Boundary}: Creates distinction between one organization and others
\item \textbf{Conveys a Sense of Identity}: Helps members understand who they are
\item \textbf{Facilitates Commitment}: Generates commitment to something larger than self-interest
\item \textbf{Enhances Stability}: Provides stability to the social system
\item \textbf{Sense-Making and Control}: Serves as sense-making mechanism and controls employee behavior
\end{enumerate}

\subsection{Characteristics of Organizational Culture}

\begin{longtable}{p{3.5cm}p{4.5cm}p{4.5cm}}
\toprule
\textbf{Characteristic} & \textbf{Description} & \textbf{Example} \\
\midrule
\textbf{Innovation \& Risk-taking} & Encourages new ideas and experimentation & DPL teams trying new AI tools \\
\textbf{Attention to Detail} & Emphasizes precision and quality & QA and code review processes \\
\textbf{Outcome Orientation} & Focus on results rather than procedures & Performance based on delivery \\
\textbf{People Orientation} & Values employee well-being and respect & Flat hierarchy, flexible work \\
\textbf{Team Orientation} & Promotes collaboration & Cross-functional product pods \\
\textbf{Aggressiveness} & Competitive and ambitious attitude & Sales pursuing bold targets \\
\textbf{Stability} & Desire for order and consistency & Corporate HR processes \\
\bottomrule
\end{longtable}

\subsection{Schein's Model of Organizational Culture (Three Levels)}

\begin{longtable}{p{3cm}p{3.5cm}p{3.5cm}p{3.5cm}}
\toprule
\textbf{Level} & \textbf{Description} & \textbf{Examples} & \textbf{Characteristics} \\
\midrule
\textbf{Artifacts and Symbols} (Visible) & The ``visible'' symbols of the culture & Logos, architecture, structure, business processes & Observable, tangible, easy to see \\
\textbf{Espoused Values} (Stated) & The company's declared set of values and norms & Mission statements, value statements & Stated beliefs, official values \\
\textbf{Basic Underlying Assumptions} (Deepest) & The bedrock of organizational culture & Deeply embedded beliefs, unconscious assumptions & Deeply embedded, often unconscious, hardest to change \\
\bottomrule
\end{longtable}

\subsection{Do Organizations Have Uniform Culture?}
\begin{itemize}
\item \textbf{Dominant Culture}: Expresses the core values shared by a majority of the organization's members
\item \textbf{Subcultures}: Minicultures within an organization, typically defined by department designations and geographical separation
\item \textbf{Core Values}: The primary or dominant values accepted throughout the organization
\item \textbf{Strong Culture}: A culture in which core values are intensely held and widely shared
\end{itemize}

\subsection{Relationship Between Structure and Culture}
\begin{itemize}
\item \textbf{How Structure Influences Culture}: Hierarchical structures $\rightarrow$ Formal culture; Flat structures $\rightarrow$ Informal culture
\item \textbf{How Culture Influences Structure}: Innovation culture $\rightarrow$ Flexible structures; Stability culture $\rightarrow$ Hierarchical structures
\item \textbf{Alignment}: Structure and culture should align; mismatch creates problems
\end{itemize}

\section{ORGANIZATIONAL COMMUNICATION (Q5 - 10 marks)}

\subsection{Grapevine Communication}
\begin{itemize}
\item \textbf{Definition}: Informal communication network
\item \textbf{Why Grapevine Thrives} (5 marks):
\begin{enumerate}
\item Response to important situations
\item Ambiguity in formal communication
\item Conditions that arouse anxiety
\item Fills information gaps
\item Creates sense of closeness and friendship
\item Fast information spread
\item Lack of formal communication
\end{enumerate}
\end{itemize}

\subsection{Managing Grapevine/Rumors (5 marks)}
\begin{enumerate}
\item \textbf{Provide information regularly} (best defense)
\item Explain actions and decisions
\item Don't shoot the messenger
\item Maintain open communication channels
\item Regular updates and transparency
\item Address concerns promptly
\item Create formal communication channels
\item Encourage employees to come forward
\end{enumerate}

\subsection{Communication Process Model}
\begin{enumerate}
\item Sender $\rightarrow$ 2. Encoding $\rightarrow$ 3. Message $\rightarrow$ 4. Channel $\rightarrow$ 5. Decoding $\rightarrow$ 6. Receiver $\rightarrow$ 7. Feedback
\end{enumerate}
\begin{itemize}
\item \textbf{Noise}: Interference (physical, psychological, semantic)
\end{itemize}

\subsection{Channel Richness}
\begin{itemize}
\item \textbf{Rich}: Face-to-face (highest), video conferencing, telephone
\item \textbf{Moderate}: Email, instant messaging, written documents
\item \textbf{Lean}: Bulletins, memos, formal reports
\end{itemize}

\subsection{Communication Networks}

\begin{longtable}{p{2.5cm}p{3.5cm}p{4cm}p{4cm}}
\toprule
\textbf{Network} & \textbf{Structure} & \textbf{Characteristics} & \textbf{Best For} \\
\midrule
\textbf{Chain} & Linear: A$\rightarrow$B$\rightarrow$C$\rightarrow$D$\rightarrow$E & Sequential, slow, high accuracy & Assembly lines \\
\textbf{Wheel} & Central hub & Fast, centralized, leader satisfaction & Strong leader, crisis \\
\textbf{All-Channel} & Everyone with everyone & Fast, high satisfaction, complex tasks & Problem-solving \\
\textbf{Circle} & Adjacent members only & Moderate speed, moderate satisfaction & Committees \\
\bottomrule
\end{longtable}

\subsection{Barriers to Communication}
\begin{itemize}
\item \textbf{Physical}: Distance, noise, poor technology, time zones
\item \textbf{Psychological}: Perceptions, emotions, attitudes, stress
\item \textbf{Language/Semantic}: Jargon, technical terms, ambiguity
\item \textbf{Cultural}: Different values, norms, styles
\item \textbf{Organizational}: Hierarchy, status, information overload
\item \textbf{Personal}: Poor listening, defensiveness, biases
\end{itemize}

\subsection{Active Listening Techniques}
\begin{itemize}
\item \textbf{Paraphrasing}: Restate in your own words
\item \textbf{Reflecting feelings}: Acknowledge emotions
\item \textbf{Asking open-ended questions}: Encourage elaboration
\item \textbf{Summarizing}: Review key points
\item \textbf{Nonverbal encouragement}: Nodding, eye contact, appropriate expressions
\end{itemize}

\section{GROUP DYNAMICS \& COMMUNICATION}

\subsection{Tuckman's Five-Stage Model}
\begin{enumerate}
\item \textbf{Forming}: High dependence, low trust, unclear roles
\item \textbf{Storming}: Conflict, tension, power struggles
\item \textbf{Norming}: Cooperation, trust building, role clarity
\item \textbf{Performing}: High performance, synergy, goal achievement
\item \textbf{Adjourning}: Reflection, transition, disbanding
\end{enumerate}

\subsection{Group Decision-Making Techniques}

\begin{longtable}{p{2.5cm}p{4cm}p{3.5cm}p{3.5cm}}
\toprule
\textbf{Technique} & \textbf{Process} & \textbf{Best For} & \textbf{Benefits} \\
\midrule
\textbf{Brainstorming} & Generate ideas without criticism & Creative solutions & Participation, innovation \\
\textbf{Nominal Group} & Silent generation $\rightarrow$ sharing $\rightarrow$ discussion $\rightarrow$ voting & Structured input needed & Prevents dominance, all voices heard \\
\textbf{Delphi} & Anonymous expert input $\rightarrow$ summary $\rightarrow$ feedback & Expert knowledge, dispersed & Avoids pressure, expert input \\
\textbf{Consensus} & All agree to support decision & High commitment needed & High buy-in, shared ownership \\
\textbf{Majority Vote} & Decision by majority rule & Fast decision needed & Quick, clear outcome \\
\textbf{Authority} & Leader makes decision & Urgent, clear accountability & Fastest, clear responsibility \\
\bottomrule
\end{longtable}

\subsection{Groupthink Symptoms}
\begin{itemize}
\item Illusion of invulnerability, collective rationalization, belief in inherent morality
\item Stereotyped views of out-groups, direct pressure on dissenters, self-censorship
\item Illusion of unanimity, mindguards
\end{itemize}

\subsection{Group Structure Elements}
\begin{enumerate}
\item \textbf{Roles}: Task roles, Maintenance roles, Individual roles
\item \textbf{Norms}: Shared expectations about behavior (explicit and implicit)
\item \textbf{Status}: Relative position or rank in group (formal and informal)
\item \textbf{Cohesiveness}: Forces keeping members in group
\item \textbf{Size}: Affects interaction, communication, decision-making
\end{enumerate}

\subsection{Advantages of Group Decision-Making}
\begin{itemize}
\item More information and knowledge
\item Diverse perspectives and alternatives
\item Better understanding and acceptance
\item Legitimacy and validity
\item Synergy through collaboration
\end{itemize}

\subsection{Disadvantages of Group Decision-Making}
\begin{itemize}
\item Time-consuming (slower than individual)
\item Groupthink: Pressure for conformity
\item Dominance by strong personalities
\item Ambiguous responsibility
\item Social loafing: Some contribute less
\end{itemize}

\section{MANAGEMENT FUNCTIONS - POLC (Q1: Q12, Q25)}

\subsection{POLC Framework}

\begin{longtable}{p{2.5cm}p{4cm}p{6cm}}
\toprule
\textbf{Function} & \textbf{Definition} & \textbf{Key Activities} \\
\midrule
\textbf{Planning} & Goals defined and strategy established & Setting goals, developing strategies, creating action plans \\
\textbf{Organizing} & Resources gathered and tasks arranged & Arranging resources, allocating tasks, creating structure \\
\textbf{Leading} & Motivating and directing people & Influencing, inspiring, guiding, communicating \\
\textbf{Controlling} & Monitoring performance and making corrections & Measuring performance, comparing to standards, taking corrective action \\
\bottomrule
\end{longtable}

\subsection{Management Levels}

\begin{longtable}{p{3.5cm}p{4cm}p{3.5cm}p{2.5cm}}
\toprule
\textbf{Level} & \textbf{Focus} & \textbf{Examples} & \textbf{Time Horizon} \\
\midrule
\textbf{Top Management} & Strategic decisions, long-term vision & CTO, VP Engineering & Long-term \\
\textbf{Middle Management} & Tactical decisions, department coordination & Engineering Manager, Product Manager & Medium-term \\
\textbf{First-Line Management} & Operational decisions, daily supervision & Team Lead, Project Manager & Short-term \\
\bottomrule
\end{longtable}

\subsection{Management Skills}
\begin{itemize}
\item \textbf{Technical Skills}: Knowledge of specific tools, technologies, processes
\item \textbf{Human Skills}: Ability to work with and through people (interpersonal)
\item \textbf{Conceptual Skills}: Ability to see big picture and think strategically
\end{itemize}

\subsection{POLC - Key Points for MCQs}
\begin{itemize}
\item \textbf{Planning}: Goals defined and strategy established (NOT monitoring employees, gathering resources, or exchanging information)
\item \textbf{Organizing}: Resources gathered and tasks arranged
\item \textbf{Leading}: Motivating and directing people
\item \textbf{Controlling}: Monitoring performance and making corrections
\item Management functions fall into: Planning, Organizing, Controlling, Leading
\end{itemize}

\section{LEADERSHIP \& MENTORSHIP}

\subsection{Leadership vs. Management}

\begin{longtable}{p{2.5cm}p{5cm}p{5cm}}
\toprule
\textbf{Aspect} & \textbf{Management} & \textbf{Leadership} \\
\midrule
\textbf{Focus} & Processes, efficiency, stability & People, vision, change \\
\textbf{Goal} & Doing things right & Doing the right things \\
\textbf{Time} & Short-term, operational & Long-term, strategic \\
\textbf{Risk} & Minimize risk & Take calculated risks \\
\textbf{Change} & Maintain stability & Drive change \\
\bottomrule
\end{longtable}

\subsection{Leadership Styles}
\begin{enumerate}
\item \textbf{Autocratic}: Leader makes decisions alone
\item \textbf{Democratic}: Leader involves team in decision-making
\item \textbf{Laissez-Faire}: Leader provides minimal guidance
\item \textbf{Transformational}: Leader inspires through vision and charisma
\item \textbf{Transactional}: Leader uses rewards/punishments
\item \textbf{Servant Leadership}: Leader focuses on serving and developing team
\item \textbf{Situational}: Leader adapts style based on situation and team maturity
\end{enumerate}

\subsection{Leadership Theories}
\begin{itemize}
\item \textbf{Trait Theory}: Leaders are born with certain traits (intelligence, charisma)
\item \textbf{Behavioral Theory}: Leadership is learned through behaviors
\item \textbf{Contingency Theory}: Effective leadership depends on matching style to situation
\item \textbf{Transformational Theory}: Leaders transform followers through inspiration
\end{itemize}

\subsection{Emotional Intelligence in Leadership}
\begin{itemize}
\item \textbf{Self-Awareness}: Understanding emotions, strengths, weaknesses
\item \textbf{Self-Regulation}: Managing emotional responses
\item \textbf{Motivation}: Intrinsic drive to achieve goals
\item \textbf{Empathy}: Understanding others' emotions and perspectives
\item \textbf{Social Skills}: Building relationships and managing social situations
\end{itemize}

\subsection{Mentorship Types}
\begin{itemize}
\item \textbf{Formal}: Structured program with assigned pairs
\item \textbf{Informal}: Natural relationship that develops organically
\item \textbf{Peer}: Colleagues at similar levels mentoring each other
\item \textbf{Reverse}: Junior mentors senior (often on new technologies)
\item \textbf{Group}: One mentor works with multiple mentees
\item \textbf{Virtual}: Relationship conducted online/remotely
\end{itemize}

\subsection{Mentorship Benefits}
\begin{itemize}
\item \textbf{For Mentees}: Career guidance, skill development, networking, confidence, faster progression
\item \textbf{For Mentors}: Personal satisfaction, coaching skills development, fresh perspectives, legacy
\item \textbf{For Organizations}: Knowledge retention, employee development, succession planning, culture building
\end{itemize}

\subsection{Mentorship vs. Coaching vs. Training}

\begin{longtable}{p{2.5cm}p{4cm}p{4cm}p{4cm}}
\toprule
\textbf{Aspect} & \textbf{Training} & \textbf{Coaching} & \textbf{Mentorship} \\
\midrule
\textbf{Duration} & Short-term & Short-term & Long-term \\
\textbf{Focus} & Specific skills & Performance improvement & Holistic development \\
\textbf{Relationship} & Instructor-student & Coach-coachee & Mentor-mentee \\
\textbf{Scope} & Skill-focused & Goal-focused & Career-focused \\
\bottomrule
\end{longtable}

\section{CONFLICT MANAGEMENT \& NEGOTIATION}

\subsection{Conflict Management Styles (Thomas-Kilmann)}

\begin{longtable}{p{2.5cm}p{2.5cm}p{3cm}p{6cm}}
\toprule
\textbf{Style} & \textbf{Assertiveness} & \textbf{Cooperativeness} & \textbf{When Appropriate} \\
\midrule
\textbf{Competing} & High & Low & Quick decision, protecting vital interests \\
\textbf{Accommodating} & Low & High & Issue more important to others, preserving harmony \\
\textbf{Avoiding} & Low & Low & Issue trivial, cooling-off needed \\
\textbf{Collaborating} & High & High & Both concerns important, commitment needed \\
\textbf{Compromising} & Moderate & Moderate & Goals moderately important, equal power \\
\bottomrule
\end{longtable}

\subsection{Types of Conflict}
\begin{itemize}
\item \textbf{Task Conflict}: Disagreement about work content/goals/methods (can be constructive)
\item \textbf{Relationship Conflict}: Personal incompatibilities, personality clashes (usually destructive)
\item \textbf{Process Conflict}: Disagreement about how work should be done (can be constructive or destructive)
\item \textbf{Value Conflict}: Fundamental differences in beliefs/principles
\end{itemize}

\subsection{Conflict Resolution Process}
\begin{enumerate}
\item Acknowledge the Conflict
\item Define the Problem
\item Gather Information
\item Generate Options
\item Evaluate and Select Solution
\item Implement and Follow Up
\end{enumerate}

\subsection{Negotiation Process}
\begin{enumerate}
\item \textbf{Preparation}: Define goals, research other party, determine BATNA, set target/reservation/opening
\item \textbf{Opening}: Establish rapport, state position, listen, identify agreement/disagreement
\item \textbf{Bargaining}: Make concessions strategically, focus on interests not positions, explore options
\item \textbf{Closing}: Summarize agreements, confirm understanding, put in writing
\item \textbf{Implementation}: Follow through, monitor progress, maintain relationship
\end{enumerate}

\subsection{BATNA (Best Alternative To a Negotiated Agreement)}
\begin{itemize}
\item Your best option if negotiations fail
\item Determines your \textbf{reservation point} (minimum acceptable outcome)
\item \textbf{Stronger BATNA = More leverage}
\end{itemize}

\subsection{Types of Negotiation}
\begin{itemize}
\item \textbf{Distributive (Zero-Sum)}: Fixed pie, one party's gain is another's loss (e.g., salary)
\item \textbf{Integrative (Win-Win)}: Expanding the pie, creating value for all (e.g., partnerships)
\end{itemize}

\section{INTERVIEW SKILLS}

\subsection{STAR Method}
\begin{itemize}
\item \textbf{S}ituation: Set the context and background
\item \textbf{T}ask: Describe what needed to be accomplished
\item \textbf{A}ction: Explain what you specifically did
\item \textbf{R}esult: Share the outcome and what you learned
\end{itemize}

\subsection{Common Interview Questions}
\begin{enumerate}
\item \textbf{``Tell me about yourself''}: 2--3 minutes, professional, relevant, connect to position
\item \textbf{``What are your strengths?''}: Relevant to job, specific examples, show benefit
\item \textbf{``What are your weaknesses?''}: Real but manageable, show growth, steps to improve
\item \textbf{``Why do you want this job?''}: Show research, connect skills/goals, alignment with values
\item \textbf{``Why should we hire you?''}: Highlight unique qualifications, match to requirements, differentiate
\item \textbf{``Tell me about a challenge''}: Use STAR, show problem-solving, learning, growth
\item \textbf{``Where do you see yourself in 5 years?''}: Show ambition but realistic, align with company growth
\item \textbf{``Do you have any questions?''}: Always have questions, show interest, ask about role/team/company
\end{enumerate}

\subsection{Body Language}
\begin{itemize}
\item \textbf{Positive}: Eye contact (60--70\%), sit up straight, natural gestures, smile appropriately, clear voice, firm handshake
\item \textbf{Negative to Avoid}: Crossing arms, avoiding eye contact, fidgeting, slouching, checking phone
\end{itemize}

\subsection{Technical Interview Preparation (IT)}
\begin{itemize}
\item Review fundamentals (programming, data structures, algorithms, system design)
\item Practice coding (LeetCode, HackerRank, CodeSignal)
\item Review your projects (be ready to discuss technical decisions, architecture, challenges)
\item Problem-solving approach: Think out loud, ask clarifying questions, start with brute force then optimize
\end{itemize}

\part{MEDIUM PRIORITY TOPICS (25 Marks)}

\section{BUSINESS MODELS \& REVENUE MODELS (Q1: Q1, Q20, Q24)}

\subsection{Key Definitions}
\begin{itemize}
\item \textbf{Revenue Model}: Framework for generating income (NOT same as business model)
\item \textbf{Business Model}: Overall approach to creating, delivering, and capturing value
\item \textbf{Value Proposition}: Promise of value to be delivered to customer (NOT same as business model or revenue model)
\end{itemize}

\subsection{Common Business Models}

\begin{longtable}{p{2.5cm}p{5cm}p{4cm}}
\toprule
\textbf{Model} & \textbf{Description} & \textbf{Example} \\
\midrule
\textbf{Freemium} & Free basic tier, paid premium features & Dropbox, Spotify \\
\textbf{Donationware} & Software supported by donations & Open source projects \\
\textbf{Crippleware} & Limited version to encourage purchase & Trial software \\
\textbf{Subscription/SaaS} & Recurring monthly/annual fees & Microsoft 365, Salesforce \\
\textbf{Advertising-Supported} & Free to users, revenue from ads & Google, Facebook \\
\textbf{Marketplace/Platform} & Connects buyers/sellers, commission & Amazon, Uber \\
\textbf{Usage-Based} & Pay per use (API calls, compute) & AWS, Twilio \\
\textbf{Product License} & One-time software sale & Traditional enterprise software \\
\textbf{Licensing \& White-Label} & License technology or provide rebrandable solutions & B2B software licensing \\
\textbf{Open Source \& Dual Licensing} & Free core code + paid support/hosting/commercial license & MySQL, MongoDB \\
\textbf{Data-Driven/Analytics} & Sell insights or data products & Analytics platforms \\
\bottomrule
\end{longtable}

\subsection{Key Distinctions (Important for MCQs)}
\begin{itemize}
\item \textbf{Revenue Model $\neq$ Business Model}: Revenue model is framework for generating income; business model is overall approach to creating/delivering/capturing value
\item \textbf{Value Proposition $\neq$ Business Model $\neq$ Revenue Model}: Value proposition is promise of value to customer
\item \textbf{Freemium}: Some features free, advanced features require premium/payment (NOT commission model, donationware, or crippleware)
\end{itemize}

\subsection{Business Model Canvas (Key Blocks)}
\begin{itemize}
\item \textbf{Customer Segments}: Who you serve (B2C, B2B, B2G, multi-sided)
\item \textbf{Value Proposition}: Which problem you solve and why you're better
\item \textbf{Channels}: How you reach customers (web, app stores, partners, sales)
\item \textbf{Customer Relationships}: Self-service, communities, account managers
\item \textbf{Revenue Streams}: Subscriptions, ads, fees, licenses, services
\item \textbf{Key Resources}: Team, tech, IP, brand, data
\item \textbf{Key Activities}: Building, operating, marketing, supporting the product
\item \textbf{Key Partners}: Cloud providers, payment processors, resellers, integrators
\item \textbf{Cost Structure}: Salaries, infra, marketing, support, compliance costs
\end{itemize}

\subsection{Horizontal vs. Vertical IT}
\begin{itemize}
\item \textbf{Horizontal}: Generic tools for many industries (email, office suites, generic CRM, infrastructure)
\item \textbf{Vertical}: Domain-specific solutions (core banking, HIS, airline booking, LMS)
\end{itemize}

\section{IT INDUSTRY VERTICALS}

\subsection{Major Verticals \& IT Needs}

\begin{longtable}{p{2.5cm}p{5cm}p{5cm}}
\toprule
\textbf{Vertical} & \textbf{Key Systems} & \textbf{Priorities} \\
\midrule
\textbf{BFSI} & Core banking, payment gateways, trading platforms, fraud/AML/KYC & High security, uptime (99.99\%+), regulatory compliance, auditability \\
\textbf{Healthcare} & HIS, EHR/EMR, LIS, telemedicine & Patient privacy (PHI), accuracy, device integration, HL7/FHIR standards \\
\textbf{Telecom} & Billing/charging, CRM, network management, provisioning & Massive event volumes, real-time rating, scalability, lawful intercept \\
\textbf{Retail/E-Commerce} & POS, inventory, e-commerce platforms, recommendations & UX, checkout speed, seasonal scaling, payment/logistics integration \\
\textbf{Manufacturing} & ERP, MRP, MES, SCADA, supply chain & Real-time monitoring, optimization, machine integration (IoT) \\
\textbf{Government} & Citizen ID (NADRA), tax/e-filing, land records, e-governance & Scalability, transparency, security, privacy vs. FOI balance \\
\textbf{Education} & SIS, LMS, exam platforms, virtual classrooms & Secure exams, scalable content, student records, accessibility \\
\textbf{Energy/Utilities} & Smart metering, grid monitoring (SCADA), outage management & Reliability, real-time telemetry, cyber-security of critical infrastructure \\
\textbf{Media/Entertainment} & Streaming platforms, DRM, recommendation engines, game servers & Low-latency delivery, peak-traffic scaling, IP/copyright management \\
\bottomrule
\end{longtable}

\subsection{Cross-Cutting Concerns Across Verticals}
\begin{itemize}
\item \textbf{Security \& Compliance}: Finance (KYC/AML), healthcare (PHI), telecom (intercept), government (FOI + privacy)
\item \textbf{Data \& Analytics}: Fraud detection, personalization, operational optimization
\item \textbf{Cloud \& APIs}: Many vertical products delivered as SaaS/microservices with integration APIs
\item \textbf{Domain Knowledge}: As important as technical skill for roles like BA, architect, PO in a given vertical
\end{itemize}

\section{AGILE SOFTWARE DEVELOPMENT}

\subsection{Agile Manifesto - Four Values}
\begin{enumerate}
\item \textbf{Individuals and interactions} over processes and tools
\item \textbf{Working software} over comprehensive documentation
\item \textbf{Customer collaboration} over contract negotiation
\item \textbf{Responding to change} over following a plan
\end{enumerate}

\subsection{Core Principles}
\begin{itemize}
\item Deliver working software frequently (weeks rather than months)
\item Welcome changing requirements (even late in development)
\item Business people and developers work together daily
\item Build projects around motivated individuals
\item Face-to-face conversation is most effective
\item Working software is primary measure of progress
\item Sustainable development pace
\item Continuous attention to technical excellence
\item Simplicity---maximizing work not done
\item Self-organizing teams
\item Regular reflection and adjustment (retrospectives)
\end{itemize}

\subsection{Scrum Overview}

\textbf{Roles:}
\begin{itemize}
\item \textbf{Product Owner}: Owns backlog, prioritizes by value, represents stakeholders
\item \textbf{Scrum Master}: Facilitates process, removes impediments, protects team
\item \textbf{Development Team}: Cross-functional, self-organizing, delivers increments
\end{itemize}

\textbf{Events:}
\begin{itemize}
\item \textbf{Sprint} (1--4 weeks): Timeboxed iteration, potentially shippable increment
\item \textbf{Sprint Planning}: Team selects backlog items, plans work
\item \textbf{Daily Scrum}: 15-minute daily meeting (What did I do? What will I do? Impediments?)
\item \textbf{Sprint Review}: Demo of completed work, stakeholder feedback
\item \textbf{Sprint Retrospective}: Team reflects on process, identifies improvements
\end{itemize}

\textbf{Artifacts:}
\begin{itemize}
\item \textbf{Product Backlog}: Ordered list of all work
\item \textbf{Sprint Backlog}: Selected items for sprint
\item \textbf{Increment}: Sum of completed items (must meet Definition of Done)
\end{itemize}

\subsection{XP (Extreme Programming) Practices}
\begin{itemize}
\item \textbf{TDD} (Test-Driven Development): Write tests first
\item \textbf{Pair Programming}: Two developers work together
\item \textbf{Continuous Integration}: Frequent code integration and testing
\item \textbf{Refactoring}: Continuously improve code structure
\item \textbf{Simple Design}: Design for current needs
\item \textbf{Collective Code Ownership}: Anyone can modify any code
\end{itemize}

\subsection{Estimation \& Planning}
\begin{itemize}
\item \textbf{Story Points}: Relative estimation (Fibonacci: 1, 2, 3, 5, 8, 13, 21)
\item \textbf{Velocity}: Story points finished per sprint (used for forecasting)
\item \textbf{User Stories}: Capture requirements from user's perspective
\item \textbf{Product Backlog}: Prioritized by value/risk; refined continuously
\end{itemize}

\subsection{Benefits and Challenges}
\begin{itemize}
\item \textbf{Benefits}: Faster, incremental value delivery; better alignment with real user needs; higher transparency and adaptability; continuous improvement
\item \textbf{Challenges}: Cultural change; risk of misinterpreting Agile as ``no planning/documentation''; coordination harder in large or highly regulated environments; needs discipline and good engineering
\end{itemize}

\section{COMPUTER CRIMES \& PECA (Q1: Q17, Q18, Q19 | Q4: Q4b, Q4c)}

\subsection{Computer Crime Definition}
\begin{itemize}
\item Unlawful activity where computer is \textbf{tool, target, or storage medium}
\end{itemize}

\subsection{Major Categories}
\begin{itemize}
\item \textbf{Unauthorized access \& hacking}: Password attacks, SQLi, XSS, session hijacking, privilege escalation
\item \textbf{Data crimes}: Theft, copying, leaks, alteration, destruction
\item \textbf{Malware \& ransomware}: Viruses, worms, trojans, spyware, encryption + ransom
\item \textbf{Online fraud}: Phishing, scams, crypto fraud, investment scams
\item \textbf{Identity theft}: Misuse of CNIC, SIM, email, social profiles
\item \textbf{Harassment \& cyberstalking}: Threats, monitoring, leaking private images
\item \textbf{Content offences}: Revenge porn, CSAM, hate speech, extremist content
\item \textbf{Cyber terrorism}: Attacks on critical infrastructure (telecom, NADRA, banks, energy)
\item \textbf{Social engineering crimes}: Deepfakes, AI voice calls, WhatsApp/OTP scams
\end{itemize}

\subsection{Technical Attack Lifecycle \& Footprinting}
\begin{itemize}
\item \textbf{Footprinting / OSINT}: Collect domain, IP, DNS, tech stack, leaked creds, employee info, public repos/docs
\item \textbf{Passive} (no direct contact) vs \textbf{active} (scans/probes)
\item Goals: understand security posture, narrow focus, map networks, find vulnerabilities
\end{itemize}

\subsection{Business Attacks Purpose (Q4b - 3 marks)}
\begin{enumerate}
\item Disrupt operations
\item Steal data/information
\item Cause financial harm
\item Gain competitive advantage
\item Damage reputation
\end{enumerate}

\subsection{Jurisdiction Challenges (Q4c - 3 marks)}
\begin{enumerate}
\item Borderless nature (multiple countries involved)
\item Which country's laws apply?
\item Extradition challenges
\item Different legal standards
\item Anonymous attackers (VPN, TOR, crypto)
\item Cryptocurrency payments hard to trace
\item Server locations in multiple countries
\end{enumerate}

\subsection{PECA 2016 - Pakistan's Cybercrime Law}

\begin{longtable}{p{4cm}p{6cm}p{4cm}}
\toprule
\textbf{Offence Category} & \textbf{Examples} & \textbf{Penalties} \\
\midrule
\textbf{Harassment \& Personal Harm} & Cyberstalking, non-consensual images, explicit content (especially minors) & 3--7 years, Rs. 1--10M \\
\textbf{Fraud, Forgery \& Identity} & Electronic fraud, identity theft, spoofing, spam, hacking tools & Months to 7 years, tens of thousands to millions \\
\textbf{System, Data \& Critical Infrastructure} & Unauthorized access, data theft, malware, interception, cyber terrorism & Up to 14 years, Rs. 50M (cyber terrorism) \\
\bottomrule
\end{longtable}

\subsection{Criticism of PECA}
\begin{itemize}
\item Claimed to be \textbf{over-broad and harsh}, with overlapping offences and disproportionate punishments
\item Vague wording and wide powers may allow misuse against journalists, activists and ordinary users
\item Threatens \textbf{freedom of expression} and access to information
\item Surveillance criteria and content-blocking powers seen as too open-ended
\item Law mixes \textbf{cybercrime}, \textbf{cyberterrorism} and \textbf{cyberwarfare} in one framework
\end{itemize}

\subsection{Data Mining \& Related Concepts}
\begin{itemize}
\item \textbf{Data Mining}: Process of analyzing data to discover patterns
\item \textbf{Collaborative Filtering}: Form of data mining (used in recommendations, NOT credit reports or flash cookies)
\item \textbf{Micro Targeting}: Using data mining to target specific groups (e.g., political campaigns determine voters most likely to support particular candidate)
\end{itemize}

\section{PROFESSIONAL LICENSING (Q1: Q23 | Q4: Q4a)}

\subsection{Licensing Definition}
\begin{itemize}
\item Process by which candidates evaluated to determine readiness to enter profession
\end{itemize}

\subsection{Purpose of Licensing (Q4a - 3 marks)}
\begin{enumerate}
\item Ensure minimum competency
\item Protect public interest
\item Regulate profession
\item Maintain standards
\item Legal authorization to practice
\end{enumerate}

\subsection{Related Concepts}
\begin{itemize}
\item \textbf{Certification}: Voluntary recognition of skills/knowledge
\item \textbf{Accreditation}: Recognition of educational institutions/programs
\end{itemize}

\section{CONTRACTS (Q4: Q4d, Q4e)}

\subsection{Indemnity (Q4d - 3 marks)}
\begin{itemize}
\item \textbf{Definition}: Contractual obligation to compensate for loss or damage
\item Protection against liability
\item Financial compensation
\item Risk transfer mechanism
\end{itemize}

\subsection{Cost-Plus vs. Fixed Price Contracts (Q4e - 3 marks)}

\begin{longtable}{p{2.5cm}p{5cm}p{3cm}p{3cm}}
\toprule
\textbf{Type} & \textbf{Description} & \textbf{Risk} & \textbf{Example} \\
\midrule
\textbf{Cost-Plus} & Pay actual costs + profit margin & Client bears cost overruns & Software dev: hours + overhead + profit \% \\
\textbf{Fixed Price} & Agreed price regardless of actual costs & Contractor bears cost overruns & Software project: \$50,000 fixed price \\
\bottomrule
\end{longtable}

\subsection{Employment \& Legal Issues}
\begin{itemize}
\item \textbf{Firing Employees}: Ethical and legal if employee doesn't perform according to expectation or fails to follow contractual obligations
\item \textbf{Contractual Obligations}: Employees must fulfill their contractual duties; failure to do so can result in termination
\end{itemize}

\section{LEGAL SYSTEM OF PAKISTAN (Q1: Q8, Q13, Q14, Q15)}

\subsection{Key Definitions}
\begin{itemize}
\item \textbf{Jurisdiction}: Area covered by single legal system and set of laws
\item \textbf{Succession}: Falls under \textbf{Civil Law}
\item \textbf{Court System}: 
\begin{itemize}
\item Civil courts fall directly under High Courts (\textbf{True})
\item Hierarchy: Supreme Court $\rightarrow$ High Courts $\rightarrow$ Civil Courts
\end{itemize}
\item \textbf{Legislative Process}: National Assembly and Senate (NOT Superior judiciary or three pillars)
\end{itemize}

\subsection{Court Hierarchy}
\begin{enumerate}
\item \textbf{Supreme Court of Pakistan} -- Highest court, constitutional interpretation, final appeals
\item \textbf{High Courts} (one per province + Islamabad) -- Appeals from lower courts, some original cases
\item \textbf{District \& Sessions Courts} -- Civil and criminal cases at district level
\item \textbf{Special Courts/Tribunals} -- Banking courts, taxation, anti-terrorism, cybercrime/PECA courts
\end{enumerate}

\subsection{Legal System - Key Points for MCQs}
\begin{itemize}
\item \textbf{Jurisdiction}: Area covered by single legal system and set of laws (NOT Executive, Legislature, or Torts)
\item \textbf{Succession}: Falls under Civil Law (NOT Criminal Law, Public Law, or Substantive Law)
\item \textbf{Civil Courts}: Fall directly under High Courts (True) -- NOT under Supreme Court directly
\item \textbf{Legislative Process}: National Assembly and Senate (NOT Superior judiciary or three pillars of state)
\end{itemize}

\section{INTELLECTUAL PROPERTY (Q1: Q6)}

\subsection{IP Types}

\begin{longtable}{p{2.5cm}p{4cm}p{3.5cm}p{4cm}}
\toprule
\textbf{Type} & \textbf{Protects} & \textbf{Duration (Pakistan)} & \textbf{Example} \\
\midrule
\textbf{Copyright} & Expression (code, UI, docs, media) & Life + 50 years & Software code, documentation \\
\textbf{Trademark} & Brand identifiers (names, logos) & Renewable & App name, company logo \\
\textbf{Patent} & New, inventive, industrially applicable inventions & 20 years & Novel technical processes \\
\textbf{Trade Secret} & Confidential business information & As long as secret kept & Proprietary algorithms, AI models \\
\bottomrule
\end{longtable}

\subsection{Key Points}
\begin{itemize}
\item \textbf{Trademark and Domain Names}: Trademark can be used as domain name (Yes), but should be used carefully to avoid conflicts
\item \textbf{Copyright Violation}: Making unauthorized copies of copyrighted software is NOT ethical and NOT legal
\item \textbf{Planting Viruses}: Planting viruses in someone else's computer is NOT ethical and NOT legal
\item \textbf{Copyright protects expression, NOT ideas}
\end{itemize}

\subsection{Protecting a Startup/FYP}
\begin{itemize}
\item Use \textbf{copyright} automatically for code and content
\item Consider \textbf{patent} if truly novel technical solution
\item Keep some elements as \textbf{trade secrets} (architecture, tuning)
\item Register \textbf{trademark} early for product/company name and logo
\end{itemize}

\subsection{Pakistan's IP Framework}
\begin{itemize}
\item \textbf{Copyright Ordinance 1962}
\item \textbf{Patents Ordinance 2000}
\item \textbf{Trade Marks Ordinance 2001}
\item \textbf{Registered Designs Ordinance 2000}
\item Administered by \textbf{IPO-Pakistan}
\end{itemize}

\subsection{International Protection}
\begin{itemize}
\item \textbf{Copyright}: Automatic globally (Berne Convention)
\item \textbf{Patents}: File in each country or via \textbf{PCT}; no global patent
\item \textbf{Trademarks}: National filings or \textbf{Madrid Protocol}
\item \textbf{Trade Secrets}: Protected via contracts and secrecy, not registration
\end{itemize}

\section{PRIVACY, DATA PROTECTION \& FOI}

\subsection{Core Data Protection Principles}
\begin{enumerate}
\item \textbf{Fair \& lawful processing} -- Clear, honest reasons for collection
\item \textbf{Purpose limitation} -- Use only for stated purposes
\item \textbf{Data minimization} -- Collect only necessary data
\item \textbf{Accuracy} -- Keep data correct and updateable
\item \textbf{Storage limitation} -- Define retention, delete when no longer needed
\end{enumerate}

\subsection{Security Measures}
\begin{itemize}
\item \textbf{Access control}: RBAC, least privilege, MFA
\item \textbf{Encryption}: HTTPS/TLS in transit; encrypted databases/backups at rest
\item \textbf{Integrity \& verification}: Hashes, signatures, change logs, audit trails
\item \textbf{Backup \& recovery}: Regular backups, off-site redundancy, disaster recovery
\item \textbf{Monitoring \& breach response}: Intrusion detection, anomaly alerts, 72-hour notification
\end{itemize}

\subsection{GDPR (EU) - Key Rights}
\begin{itemize}
\item Informed consent, access, correction, deletion (``right to be forgotten''), portability
\item \textbf{72-hour breach notification}
\item Fines: Up to \texteuro{}20M or 4\% of global turnover
\end{itemize}

\subsection{Other Privacy Laws}
\begin{itemize}
\item \textbf{HIPAA (US health)}: Protects PHI (Protected Health Information)
\item \textbf{CCPA (California)}: Rights to know, delete, opt-out of data selling
\item \textbf{UAE PDPL / Saudi PDPL}: Strict consent, security, cross-border transfer rules
\end{itemize}

\subsection{Article 19A \& FOI (Pakistan)}
\begin{itemize}
\item \textbf{Article 19A}: Right to access information on matters of public importance
\item \textbf{FOI/RTI laws}: Citizens can request public records (spending, contracts, policies)
\item \textbf{Not accessible}: National security/classified info, private citizens' personal data, internal memos
\end{itemize}

\subsection{Privacy by Design Principles}
\begin{enumerate}
\item Build privacy into architecture from start
\item Default to privacy (opt-in tracking, conservative permissions)
\item Be transparent (clear notices, easy-to-use privacy settings)
\item Limit internal access (RBAC, no blanket production access)
\item Secure data end-to-end (encryption, monitoring, deletion policies)
\item Test regularly (privacy impact assessments, pen-tests)
\end{enumerate}

\subsection{Data Mining \& Re-identification}
\begin{itemize}
\item \textbf{Data Mining}: ML/statistical analysis for patterns (recommendations, fraud detection)
\item \textbf{Secondary Use}: Using data for new purposes not originally disclosed $\rightarrow$ often illegal
\item \textbf{Re-identification}: ``Anonymous'' datasets can be linked to other data to identify individuals (Netflix Prize case)
\end{itemize}

\part{SUPPORTING TOPICS}

\section{EXAM STRATEGY TIPS}

\subsection{Time Allocation}
\begin{itemize}
\item \textbf{Q1 (MCQ)}: $\sim$1 minute per question (25 minutes total)
\item \textbf{Q2 (Ethical Decision Making)}: $\sim$45 minutes
\item \textbf{Q3 (Business Structures)}: $\sim$20 minutes
\item \textbf{Q4 (Short Answers)}: $\sim$20 minutes (4 minutes per question)
\item \textbf{Q5 (Communication)}: $\sim$15 minutes
\item \textbf{Q6 (Structure Design)}: $\sim$15 minutes
\end{itemize}

\subsection{For Q2 (Ethical Decision Making - 25 marks)}
\begin{enumerate}
\item \textbf{Read scenario carefully} (5 min) -- underline key facts
\item \textbf{Step I: Understanding} (10 min) -- 10--15 facts, 4--6 ethical issues, 8--10 stakeholders
\begin{itemize}
\item Facts: Number each fact, be neutral and logical
\item Ethical issues: Why it's an issue, potential/resulting harm
\item Stakeholders: Directly and indirectly affected parties
\end{itemize}
\item \textbf{Step II: Dilemma} (2 min) -- frame clear either/or question
\begin{itemize}
\item Format: ``Should [person/entity] [action] or [alternative action]?''
\end{itemize}
\item \textbf{Step III: Analysis} (15 min) -- Use ALL THREE frameworks (A--G, 1--5, H--P) for BOTH alternatives
\begin{itemize}
\item Consequentialism (A--G): Harm/benefit analysis for both alternatives
\item Rights and Duties (1--5): Rights, duties, precedence for both alternatives
\item Deontology (H--P): Respect, universalization for both alternatives
\end{itemize}
\item \textbf{Step IV: Decision} (13 min) -- Clear decision, 5--8 implementation steps, stakeholder impacts, 4--6 preventive measures, pivot point
\begin{itemize}
\item Reference frameworks (A--G, 1--5, H--P) to justify decision
\item Show which rights/duties take precedence
\end{itemize}
\end{enumerate}

\subsection{Common Mistakes to Avoid}
\begin{itemize}
\item Missing facts or stakeholders
\item Not using all three frameworks
\item Vague implementation steps
\item Forgetting to analyze BOTH alternatives
\item Not justifying which rights/duties take precedence
\item Skipping the universalization test in Deontology
\end{itemize}

\subsection{What Gets Full Marks}
\begin{itemize}
\item Complete fact list (10+ facts)
\item All stakeholders identified
\item All three frameworks used thoroughly
\item Clear, defensible decision
\item Specific, actionable implementation steps
\item Comprehensive stakeholder impact analysis
\item Multiple preventive measures across categories
\item Clear pivot point prevention
\end{itemize}

\subsection{Example Scenario: Qaswa Corporation (E-Mail-Based Effort to Boost Morale)}
\begin{itemize}
\item \textbf{Key Elements}: Employee morale, anonymous memo, system security bypass, management response
\item \textbf{Typical Stakeholders}: Khalid (employee), Maryam (colleague), Shahzeb (director), Qaswa Corporation, other employees, management, board of directors
\item \textbf{Typical Ethical Issues}: Anonymity, system security bypass, disrespect to management, impact on morale, potential harm to company reputation
\item \textbf{Typical Dilemma}: Should Shahzeb fire Khalid or take alternative disciplinary action?
\end{itemize}

\subsection{For Q3 (Business Structures - 15 marks)}
\begin{itemize}
\item Define each structure clearly
\item Compare characteristics side-by-side
\item Include advantages and disadvantages
\item Time: $\sim$20 minutes
\end{itemize}

\subsection{For Q4 (Short Answers - 15 marks)}
\begin{itemize}
\item Be concise but complete
\item 3 marks = 3--4 key points
\item Provide examples where relevant
\item Time: $\sim$20 minutes (4 minutes per question)
\end{itemize}

\subsection{For Q5 (Communication - 10 marks)}
\begin{itemize}
\item List multiple reasons/strategies
\item Be specific and practical
\item Time: $\sim$15 minutes
\end{itemize}

\subsection{For Q6 (Structure Design - 10 marks)}
\begin{itemize}
\item Analyze the scenario carefully (products, geography, centralized functions)
\item Identify key factors: strategy, size, technology, environment, geography
\item Suggest suitable structure (often Matrix or Hybrid for complex scenarios)
\item Justify your choice with factors
\item Draw clear organization chart showing reporting relationships
\item Consider: Product divisions, geographic divisions, centralized departments (Technology, PMO, etc.)
\item \textbf{For Kangaroo Corp type scenarios}: 
\begin{itemize}
\item Product-based divisions (FB, FP, FA)
\item Geographic sub-offices (country-level operations)
\item Centralized functions (development, customer services, Technology, PMO)
\item Matrix structure allows dual reporting (product + geography + function)
\end{itemize}
\item Time: $\sim$15 minutes
\end{itemize}

\section{FINAL PREPARATION CHECKLIST}

\subsection{One Week Before Exam:}
\begin{itemize}
\item[$\square$] Review all ethical theories and frameworks
\item[$\square$] Practice 4-step ethical decision making with scenarios
\item[$\square$] Memorize business structure definitions and comparisons
\item[$\square$] Review POLC framework
\item[$\square$] Study grapevine communication and management strategies
\item[$\square$] Review legal system of Pakistan
\item[$\square$] Memorize key definitions (revenue model, value proposition, etc.)
\end{itemize}

\subsection{Two Days Before Exam:}
\begin{itemize}
\item[$\square$] Practice drawing organization charts
\item[$\square$] Review all MCQ topics
\item[$\square$] Practice short answer questions
\item[$\square$] Review case studies
\end{itemize}

\subsection{Day Before Exam:}
\begin{itemize}
\item[$\square$] Quick review of all key definitions
\item[$\square$] Review 4-step ethical decision making process
\item[$\square$] Review business structures comparison
\item[$\square$] Get good sleep!
\end{itemize}

\subsection{Exam Day:}
\begin{itemize}
\item[$\square$] Read all questions first
\item[$\square$] Allocate time per question
\item[$\square$] Answer MCQs first (quick points)
\item[$\square$] Then tackle longer questions
\item[$\square$] Review answers if time permits
\end{itemize}

\section{QUICK REFERENCE CHECKLIST}

\subsection{Ethics (Must Know)}
\begin{itemize}
\item[$\square$] Utilitarianism: ``Greatest good for greatest number'' -- consequentialist
\item[$\square$] Deontology: Duty-based, Kant -- focuses on rules/duties
\item[$\square$] Egoism: Self-interest based (NOT duty-based, NOT altruism)
\item[$\square$] Hedonism: Pursuit of happiness as highest aim
\item[$\square$] 4-Step Ethical Decision Making Process (all steps)
\item[$\square$] Consequentialism, Rights and Duties, Deontology frameworks
\item[$\square$] Whistleblowers: Make unauthorized disclosures about harmful situations or fraud
\item[$\square$] Moral principles $\neq$ Law (False that they always align)
\item[$\square$] Code of conduct objectives (Discipline, Inspiration, Education -- NOT Enforcement)
\item[$\square$] Ethical theories are NOT formulas but frameworks for analysis
\item[$\square$] ACM Code: Comprehensive evaluations of systems and impacts required
\end{itemize}

\subsection{Business Structures (Must Know)}
\begin{itemize}
\item[$\square$] Veil of Incorporation definition and purpose
\item[$\square$] Sole Proprietorship characteristics
\item[$\square$] Partnership/AOP characteristics
\item[$\square$] Public Limited Company characteristics and requirements
\item[$\square$] SMC Pvt. Ltd. characteristics and requirements
\item[$\square$] Private Limited Company characteristics
\item[$\square$] Non-compete clause definition
\item[$\square$] Non-solicitation clause definition
\end{itemize}

\subsection{Management (Must Know)}
\begin{itemize}
\item[$\square$] POLC: Planning, Organizing, Leading, Controlling
\item[$\square$] Planning: Goals defined and strategy established
\item[$\square$] Organizing: Resources gathered and tasks arranged
\item[$\square$] Leading: Motivating and directing people
\item[$\square$] Controlling: Monitoring performance and making corrections
\end{itemize}

\subsection{Communication (Must Know)}
\begin{itemize}
\item[$\square$] Grapevine communication definition
\item[$\square$] Why grapevine thrives (4--5 reasons)
\item[$\square$] How to manage rumors (4--5 strategies)
\item[$\square$] Communication process model (7 steps)
\item[$\square$] Channel richness (rich, moderate, lean)
\item[$\square$] Communication networks (chain, wheel, all-channel, circle)
\end{itemize}

\subsection{Legal (Must Know)}
\begin{itemize}
\item[$\square$] Jurisdiction definition
\item[$\square$] Succession falls under Civil Law
\item[$\square$] Civil courts fall under High Courts (True)
\item[$\square$] Legislative process: National Assembly and Senate
\item[$\square$] Copyright protects expression, NOT ideas
\item[$\square$] Trademark can be used as domain name (Yes)
\item[$\square$] Making unauthorized copies is NOT ethical and NOT legal
\end{itemize}

\subsection{Business Models (Must Know)}
\begin{itemize}
\item[$\square$] Revenue model: Framework for generating income
\item[$\square$] Value proposition: Promise of value to customer
\item[$\square$] Business model: Overall approach to creating/delivering/capturing value
\item[$\square$] Freemium: Free features + premium paid features
\item[$\square$] SaaS: Subscription-based software
\item[$\square$] Marketplace: Connects buyers/sellers, earns commission
\end{itemize}

\subsection{Computer Crimes (Must Know)}
\begin{itemize}
\item[$\square$] Computer crime: Tool, target, or storage medium
\item[$\square$] Collaborative filtering: Form of data mining
\item[$\square$] Micro targeting: Using data mining for targeting
\item[$\square$] Business attacks purpose (5 points)
\item[$\square$] Jurisdiction challenges (5--6 points)
\item[$\square$] PECA 2016 offence categories and penalties
\end{itemize}

\subsection{Professional Practices (Must Know)}
\begin{itemize}
\item[$\square$] Licensing: Evaluation for profession readiness
\item[$\square$] Licensing purpose (5 points)
\item[$\square$] Indemnity: Compensation for loss/damage
\item[$\square$] Cost-plus vs. Fixed price contracts
\item[$\square$] Firing employees: Ethical and legal if they don't perform or fail contractual obligations
\item[$\square$] Unauthorized copying/planting viruses: NOT ethical and NOT legal
\end{itemize}

\subsection{Organizational Structure (Must Know)}
\begin{itemize}
\item[$\square$] Types: Simple, Functional, Divisional, Matrix, Flat, Hierarchical
\item[$\square$] Elements: Work Specialization, Departmentalization, Chain of Command, Span of Control, Centralization, Formalization
\item[$\square$] Factors: Strategy, Size, Technology, Environment, Culture, Geography, Legal
\item[$\square$] How to design and justify structure
\item[$\square$] Tall vs. Flat organizations
\end{itemize}

\subsection{Agile (Must Know)}
\begin{itemize}
\item[$\square$] Agile Manifesto four values
\item[$\square$] Scrum roles: Product Owner, Scrum Master, Development Team
\item[$\square$] Scrum events: Sprint, Sprint Planning, Daily Scrum, Sprint Review, Sprint Retrospective
\item[$\square$] Scrum artifacts: Product Backlog, Sprint Backlog, Increment
\item[$\square$] XP practices: TDD, Pair Programming, Continuous Integration
\item[$\square$] Story points and velocity
\end{itemize}

\section{KEY FORMULAS/DEFINITIONS TO MEMORIZE}

\subsection{Ethics:}
\begin{itemize}
\item \textbf{Utilitarianism} = Greatest good for greatest number (consequentialist)
\item \textbf{Deontology} = Duty-based, Kant, focus on rules not outcomes
\item \textbf{Egoism} = Self-interest based
\item \textbf{Hedonism} = Pursuit of happiness as highest aim
\item \textbf{Rights \& Duties} = Individual rights + moral duties
\end{itemize}

\subsection{Business:}
\begin{itemize}
\item \textbf{Revenue Model} = Framework for generating income
\item \textbf{Value Proposition} = Promise of value to customer
\item \textbf{Business Model} = Overall approach to creating/delivering/capturing value
\item \textbf{Freemium} = Free features + premium paid features
\end{itemize}

\subsection{Legal:}
\begin{itemize}
\item \textbf{Jurisdiction} = Area covered by single legal system
\item \textbf{Veil of Incorporation} = Separation of company from owners (limited liability)
\item \textbf{Copyright} = Protects expression, NOT ideas
\item \textbf{Patent} = Protects new, inventive, industrially applicable inventions
\end{itemize}

\subsection{Management:}
\begin{itemize}
\item \textbf{POLC} = Planning, Organizing, Leading, Controlling
\item \textbf{Planning} = Goals defined and strategy established
\item \textbf{Organizing} = Resources gathered and tasks arranged
\item \textbf{Leading} = Motivating and directing people
\item \textbf{Controlling} = Monitoring performance and making corrections
\end{itemize}

\subsection{Communication:}
\begin{itemize}
\item \textbf{Grapevine} = Informal communication network
\item Thrives due to: Importance, ambiguity, anxiety, information gaps
\item \textbf{Channel Richness} = Information-carrying capacity (rich, moderate, lean)
\end{itemize}

\section{EXAM-SPECIFIC ANSWER PATTERNS \& CLARIFICATIONS}

\subsection{MCQ Answer Patterns (Based on Exam Questions)}

\subsubsection{Ethics Questions}
\begin{itemize}
\item \textbf{Q2}: Hedonism (pursuit of happiness as highest ethical aim)
\item \textbf{Q3}: Whistleblowers make unauthorized disclosures about harmful situations or fraud
\item \textbf{Q4}: Moral principles are NOT always aligned with law (False)
\item \textbf{Q5}: Statements 1 and 3 are correct (Utilitarianism is consequentialist, Deontology is Kant's theory); Statement 2 is FALSE (Egoism is NOT duty-based)
\item \textbf{Q7}: Utilitarianism (``greatest good for greatest number'')
\item \textbf{Q9}: Ethical theories are NOT formulas to solve problems
\end{itemize}

\subsubsection{Business \& Legal Questions}
\begin{itemize}
\item \textbf{Q1}: Revenue model is a framework for generating income (NOT same as business model)
\item \textbf{Q6}: Trademark can be used as domain name (Yes), but should be used carefully
\item \textbf{Q8}: Jurisdiction means area covered by single legal system
\item \textbf{Q13}: Succession falls under Civil Law
\item \textbf{Q14}: Civil courts fall directly under High Courts (True)
\item \textbf{Q15}: Legislative process consists of National Assembly and Senate
\item \textbf{Q16}: Firing employees who don't perform is Ethical and Legal
\item \textbf{Q19}: Making unauthorized copies/planting viruses is NOT ethical and NOT legal
\item \textbf{Q20}: Freemium (free features + premium payment)
\item \textbf{Q21}: Non-compete clause prevents current AND former employees from working for competitors
\item \textbf{Q22}: Non-solicitation clause prevents encouraging employees AND customers to move
\item \textbf{Q23}: Licensing is process to evaluate readiness to enter profession
\item \textbf{Q24}: Value proposition is promise of value to customer (NOT business model or revenue model)
\item \textbf{Q25}: Management functions: Planning, Organizing, Controlling, Leading
\end{itemize}

\subsubsection{Data \& Technology Questions}
\begin{itemize}
\item \textbf{Q17}: Collaborative filtering is a form of data mining
\item \textbf{Q18}: Micro targeting uses data mining to determine voters most likely to support candidate
\end{itemize}

\subsubsection{Code of Conduct Questions}
\begin{itemize}
\item \textbf{Q10}: Code of conduct objectives: Discipline, Inspiration, Education (NOT Enforcement)
\item \textbf{Q11}: ACM code requires comprehensive evaluations of systems and impacts (True)
\end{itemize}

\vspace{1cm}

\textbf{Good luck with your exam preparation! Focus on understanding concepts rather than just memorizing, especially for the ethical decision-making question.}

\end{document}
